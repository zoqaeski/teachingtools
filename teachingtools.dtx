% \iffalse meta-comment
%<*internal>
\iffalse
%</internal>
%<*readme>
teachingtools - Environments and tools for teaching materials
===============================================================

Teaching materials, such as lecture notes, tutorials, and textbooks often
require worked examples and questions for students to learn from.  This package
provides a set of environments suitable to typeset question/answer pairs,
examples, and notes.

Installation
------------

The package is supplied in `dtx` format, and built with `l3build` to take
advantage of new LaTeX3 and LuaTeX features. 

The package requires LaTeX3 support as provided in the `l3kernel` and
`l3packages` bundles. Both of these are available on
[CTAN](http://www.ctan.org/) as ready-to-install zip files.  Suitable versions
are available in MiKTeX 2.9 and TeX Live 2015 (updating the relevant packages
online may be necessary).  LaTeX3, and so `teachingtools`, requires the e-TeX
extensions: these are available on all modern TeX systems.
%</readme>
%<*internal>
\fi
\def\nameofplainTeX{plain}
\ifx\fmtname\nameofplainTeX\else
  \expandafter\begingroup
\fi
%</internal>
%<*install>
\input l3docstrip.tex
\keepsilent
\askforoverwritefalse
\preamble
---------------------------------------------------------------
The teachingtools package --- Environments and tools for teaching materials
Maintained by Robbie Smith
E-mail: zoqaeski@gmail.com
Released under the LaTeX Project Public License v1.3c or later
See http://www.latex-project.org/lppl.txt
---------------------------------------------------------------

\endpreamble
\postamble
Copyright (C) 2017 by
  Robbie Smith <zoqaeski@gmail.com>

It may be distributed and/or modified under the conditions of
the LaTeX Project Public License (LPPL), either version 1.3c of
this license or (at your option) any later version.  The latest
version of this license is in the file:
   http://www.latex-project.org/lppl.txt

This work is "maintained" (as per LPPL maintenance status) by
  Robbie Smith.

This work consists of the file  teachingtools.dtx
          and the derived files teachingtools.pdf,
                                teachingtools.sty and
                                teachingtools.ins.
\endpostamble
\usedir{tex/latex/teachingtools}
\generate{
  \file{\jobname.sty}{\from{\jobname.dtx}{package}}
}
%</install>
%<install>\endbatchfile
%<*internal>
\usedir{source/latex/teachingtools}
\generate{
  \file{\jobname.ins}{\from{\jobname.dtx}{install}}
}
\nopreamble\nopostamble
\usedir{doc/latex/teachingtools}
\generate{
  \file{README.md}{\from{\jobname.dtx}{readme}}
}
\ifx\fmtname\nameofplainTeX
  \expandafter\endbatchfile
\else
  \expandafter\endgroup
\fi
%</internal>
%<*driver|package>
\RequirePackage{expl3}[2015/09/11]
\RequirePackage{xparse}
%</driver|package>
%<*driver>
\documentclass{l3doc}
%\DisableImplementation
\begin{document}
  \DocInput{\jobname.dtx}
\end{document}
%</driver>
% \fi
%\ProvideDocumentCommand\opt{m}{\texttt{#1}}
%
%\GetFileInfo{\jobname.sty}
%
%\title{^^A
%  \textsf{teachingtools} --- Environments and tools for teaching materials \thanks{^^A
%    This file describes \fileversion, last revised \filedate.^^A
%  }^^A
%}
%\author{^^A
%  Robbie Smith\thanks
%    {^^A
%      E-mail: \href{mailto:zoqaeski@gmail.com}
%      {\texttt{zoqaeski@gmail.com}}^^A
%    }^^A
%}
%\date{Released \filedate}
%
%\maketitle
%
%\changes{v0.1}{2017/11/18}{First public testing release}
%\changes{v0.2}{2017/11/22}{Added support for environment numbering prefixes}
%
%\begin{implementation}
%
%\section{Implementation}
%
%    \begin{macrocode}
%<*package>
%    \end{macrocode}
%
%    \begin{macrocode}
%<@@=teachingtools>
%    \end{macrocode}
%
%\subsection{Preliminaries}
%
% The usual preliminaries apply.
%    \begin{macrocode}
\ProvidesExplPackage {teachingtools} {2017/11/22} {0.2}
  {Environments and tools for teaching materials}
%    \end{macrocode}
%
% This package depends on recent versions of the \pkg{l3kernel}.  Versions from
% TeXLive 2016 or newer should be fine.
%    \begin{macrocode}
\@ifpackagelater { expl3 } { 2015/11/15 }
  { }
  {
    \PackageError { teachingtools } { Support~package~expl3~too~old }
      {
        You~need~to~update~your~installation~of~the~bundles~'l3kernel'~and~
        'l3packages'.\MessageBreak
        Loading~teachingtools~will~abort!
      }
    \tex_endinput:D
  }
%    \end{macrocode}
%
% Now load the support packages. \pkg{chngcntr} is required to modify counters
% while they're in use, and \pkg{enumitem} provides convenience macros for
% modifing lists.
%    \begin{macrocode}
\RequirePackage{ l3keys2e , chngcntr , enumitem }
%    \end{macrocode}
%
%\subsection{Messages and error handling}
%
% All of the error and warning messages are defined here in one place to
% simplify things.
%    \begin{macrocode}
\msg_new:nnnn { teachingtools } { unknown-option }
  { Unknown~option~'#1'. }
  {
    The~option~'#1'~is~not~known~by~teachingtools:~
    perhaps~it~is~spelled~incorrectly.
}
%    \end{macrocode}
%
%\begin{macro}{
%  \@@_error:nxx ,
%  \@@_error:nx  ,
%  \@@_error:n
%}
%\begin{macro}{\l_@@_error_bool}
% Some general-purpose functions for throwing errors are defined next.
%    \begin{macrocode}
\cs_new_protected:Npn \@@_error:nxx #1#2#3 {
  \bool_set_true:N \l_@@_error_bool
  \msg_error:nnxx { teachingtools } {#1} {#2} {#3}
}
\cs_new_protected:Npn \@@_error:nx #1#2 {
  \@@_error:nxx {#1} {#2} { }
}
\cs_new_protected:Npn \@@_error:n #1 {
  \@@_error:nxx {#1} { } { }
}
\bool_new:N \l_@@_error_bool
%    \end{macrocode}
%\end{macro}
%\end{macro}
%
%\begin{macro}{\l_@@_key_tl}
% Unknown options should raise an \pkg{teachingtools}-specific error message.
%    \begin{macrocode}
\tl_new:N  \l_@@_key_tl
\keys_define:nn { teachingtools } {
  unknown .code:n = {
    \msg_error:nnx { teachingtools } { unknown-option }
    { \exp_not:V \l_keys_key_tl }
  }
}
\AtBeginDocument {
  \keys_define:nn { teachingtools } {
    unknown .code:n = {
      \msg_error:nnx { teachingtools } { unknown-option }
      { \exp_not:V \l_keys_key_tl }
    }
  }
}
%    \end{macrocode}
%\end{macro}
%
% \subsection{Printing routines}
%
% Some general-purpose printing routines are provided here to ensure that text
% can be printed with the necessary configuration.
% 
%\begin{macro}{
%  \@@_print_question:
%}
% This function prints the \texttt{question} environment numbering string.
%    \begin{macrocode}
\cs_new:Nn \@@_print_question: {
  \tl_clear:N \l_tmpa_tl
  \bool_if:NT \l_@@_question_prefix_bool {
    \tl_put_right:Nx \l_tmpa_tl { \l_@@_question_prefix_tl }
    \tl_put_right:Nx \l_tmpa_tl { \l_@@_prefix_separator_str }
  }
  \tl_put_right:Nx \l_tmpa_tl { \l_@@_question_label_tl }
  \tl_to_str:N \l_tmpa_tl
}
%    \end{macrocode}
%\end{macro}

%\begin{macro}{
%  \@@_print_example:
%}
% This function prints the \texttt{example} environment numbering string.
%    \begin{macrocode}
\cs_new:Nn \@@_print_example: {
  \tl_clear:N \l_tmpa_tl
  \bool_if:NT \l_@@_example_prefix_bool {
    \tl_put_right:Nx \l_tmpa_tl { \l_@@_example_prefix_tl }
    \tl_put_right:Nx \l_tmpa_tl { \l_@@_prefix_separator_str }
  }
  \tl_put_right:Nx \l_tmpa_tl { \l_@@_example_label_tl }
  \tl_to_str:N \l_tmpa_tl
}
%    \end{macrocode}
%\end{macro}
%
% \subsection{Resetting counters}
%\begin{macro}{
%  \l_@@_reset_question_tl
%}
% The \opt{reset-question} option provides a means for question numbering to be reset
% at different sectioning \enquote{levels}.  The default is for numbering to be
% reset at the \enquote{section} level.
%    \begin{macrocode}
\newcounter{question}
\keys_define:nn { teachingtools } {
  reset-question .default:n = section,
  reset-question .choices:nn = {
    part, chapter, section, subsection, subsubsection, paragraph, subparagraph
  }
  {
    \counterwithin*{question}{ \tl_to_str:N \l_keys_choice_tl }
  },
  reset-question / none .code:n = {
    \counterwithout*{question}{section}
  },
}
%    \end{macrocode}
%\end{macro}
%
%\begin{macro}{
%  \l_@@_reset_example_tl
%}
% The \opt{reset-example} option provides a means for example numbering to be reset
% at different sectioning \enquote{levels}.  The default is for numbering to be
% not reset.
%    \begin{macrocode}
\newcounter{example}
\keys_define:nn { teachingtools } {
  reset-example .default:n = none,
  reset-example .choices:nn = {
    part, chapter, section, subsection, subsubsection, paragraph, subparagraph
  }
  {
    \counterwithin*{example}{ \tl_to_str:N \l_keys_choice_tl }
  },
  reset-example / none .code:n = {
    \counterwithout*{example}{section}
  },
}
%    \end{macrocode}
%\end{macro}
%
%\begin{macro} {
%  \l_@@_reset_all_tl 
%}
% The \opt{reset-all} option provides a means for environment numbering to be reset
% at different sectioning \enquote{levels}.  The default is for numbering to be
% set individually for each environment.
%    \begin{macrocode}
\keys_define:nn { teachingtools } {
  reset-all .choice:,
  reset-all / part .meta:n = {
    reset-question = part,
    reset-example = part
  },
  reset-all / chapter .meta:n = {
    reset-question = chapter,
    reset-example = chapter
  },
  reset-all / section .meta:n = {
    reset-question = section,
    reset-example = section
  },
  reset-all / subsection .meta:n = {
    reset-question = subsection,
    reset-example = subsection
  },
  reset-all / subsubsection .meta:n = {
    reset-question = subsubsection,
    reset-example = subsubsection
  },
  reset-all / paragraph .meta:n = {
    reset-question = paragraph,
    reset-example = paragraph
  },
  reset-all / subparagraph .meta:n = {
    reset-question = subparagraph,
    reset-example = subparagraph
  },
  reset-all / none .meta:n = {
    reset-question = none,
    reset-example = none
  },
  reset-all / false .meta:n = {
    reset-question,
    reset-example
  },
  reset-all .default:n = false,
  reset-all .initial:n = false,
}
%    \end{macrocode}
%\end{macro}
%
% \subsection{Formatting counters and labels}
% The label style of both environments can be set to any of the predefined
% counter styles: arabic, alphabetic, uppercase alphabetic, Roman numerals,
% uppercase Roman numerals, and footnote symbols.  The latter style is provided
% but not recommended for use.
% 
% The following features are planned but not yet implemented:
%
% \begin{itemize}
%   \item \opt{suffix}: The character to append after the counter
%   \item formatting of environment? No idea how to do that
% \end{itemize}
%
%\begin{macro}{
%  \l_@@_question_title_tl ,
%  \l_@@_example_title_tl  ,
%}
% The environment labels are built up as a token lists by appending or prefixing
% each element as necessary.
%    \begin{macrocode}
% \tl_new:N \l_@@_question_title_tl
% \tl_new:N \l_@@_example_title_tl
%    \end{macrocode}
%\end{macro}
%
% Each environment can have an optional prefix to precede the number with
% that of the specified sectioning level. Note that this implies resetting of
% the environment counter at the specified level, unless overridden by the
% \opt{reset-question}, \opt{reset-example}, or \opt{reset-all} options. The
% default is none.
%
%\begin{macro}{
%  \l_@@_prefix_separator_tl
%}
% The \opt{prefix-separator} is a character used to separate the environment
% numbering prefix from its number, if the prefix exists.
%    \begin{macrocode}
\str_new:N \l_@@_prefix_separator_str
\keys_define:nn { teachingtools } {
  prefix-separator .choice:,
  prefix-separator / dash .code:n = {
    \str_set:Nn \l_@@_prefix_separator_str { - }
  },
  prefix-separator / dot .code:n = {
    \str_set:Nn \l_@@_prefix_separator_str { . }
  },
  prefix-separator / colon .code:n = {
    \str_set:Nn \l_@@_prefix_separator_str { : }
  },
  prefix-separator / none .code:n = {
    \str_gclear:N \l_@@_prefix_separator_str
  },
  prefix-separator .default:n = none,
  prefix-separator .initial:n = none,
}
%    \end{macrocode}
%\end{macro}
%
%\begin{macro}{
%  \l_@@_question_prefix_tl   ,
%  \l_@@_question_prefix_bool
%}
% The \opt{question-prefix} sets the prefix for the \texttt{question}
% environment.
%    \begin{macrocode}
\tl_new:N \l_@@_question_prefix_tl
\bool_new:N \l_@@_question_prefix_bool
\keys_define:nn { teachingtools } {
  question-prefix .choice:, 
  question-prefix / part .code:n = {
    \tl_set:Nn \l_@@_question_prefix_tl { \arabic{part} }
    \keys_set:nn { teachingtools } { 
      reset-question = part,
      prefix-separator = dash,
    }
    \bool_set_true:N \l_@@_question_prefix_bool
  },
  question-prefix / chapter .code:n = {
    \tl_set:Nn \l_@@_question_prefix_tl { \arabic{chapter} }
    \keys_set:nn { teachingtools } { 
      reset-question = chapter,
      prefix-separator = dash,
    }
    \bool_set_true:N \l_@@_question_prefix_bool
  },
  question-prefix / section .code:n = {
    \tl_set:Nn \l_@@_question_prefix_tl { \arabic{section} }
    \keys_set:nn { teachingtools } {
      reset-question = section,
      prefix-separator = dash,
    }
    \bool_set_true:N \l_@@_question_prefix_bool
  },
  question-prefix / subsection .code:n = {
    \tl_set:Nn \l_@@_question_prefix_tl { \arabic{subsection} }
    \keys_set:nn { teachingtools } {
      reset-question = subsection,
      prefix-separator = dash,
    }
    \bool_set_true:N \l_@@_question_prefix_bool
  },
  question-prefix / subsubsection .code:n = {
    \tl_set:Nn \l_@@_question_prefix_tl { \arabic{subsubsection} }
    \keys_set:nn { teachingtools } {
      reset-question = subsubsection,
      prefix-separator = dash,
    }
    \bool_set_true:N \l_@@_question_prefix_bool
  },
  question-prefix / paragraph .code:n = {
    \tl_set:Nn \l_@@_question_prefix_tl { \arabic{paragraph} }
    \keys_set:nn { teachingtools } {
      reset-question = paragraph,
      prefix-separator = dash,
    }
    \bool_set_true:N \l_@@_question_prefix_bool
  },
  question-prefix / subparagraph .code:n = {
    \tl_set:Nn \l_@@_question_prefix_tl { \arabic{subparagraph} }
    \keys_set:nn { teachingtools } {
      reset-question = subparagraph,
      prefix-separator = dash,
    }
    \bool_set_true:N \l_@@_question_prefix_bool
  },
  question-prefix / none .code:n = {
    \tl_gclear:N \l_@@_question_prefix_tl 
  },
  question-prefix .default:n = none,
  question-prefix .initial:n = none,
}
%    \end{macrocode}
%\end{macro}
%
%\begin{macro}{
%  \l_@@_example_prefix_tl   ,
%  \l_@@_example_prefix_bool
%}
% The \opt{example-prefix} sets the prefix for the \texttt{example}
% environment.
%    \begin{macrocode}
\tl_new:N \l_@@_example_prefix_tl
\bool_new:N \l_@@_example_prefix_bool
\keys_define:nn { teachingtools } {
  example-prefix .choice:, 
  example-prefix / part .code:n = {
    \tl_set:Nn \l_@@_example_prefix_tl { \arabic{part} }
    \bool_set_true:N \l_@@_example_prefix_bool
    \keys_set:nn { teachingtools } { 
      reset-example = part,
      prefix-separator = dash,
    }
  },
  example-prefix / chapter .code:n = {
    \tl_set:Nn \l_@@_example_prefix_tl { \arabic{chapter} }
    \bool_set_true:N \l_@@_example_prefix_bool
    \keys_set:nn { teachingtools } { 
      reset-example = chapter,
      prefix-separator = dash,
    }
  },
  example-prefix / section .code:n = {
    \tl_set:Nn \l_@@_example_prefix_tl { \arabic{section} }
    \bool_set_true:N \l_@@_example_prefix_bool
    \keys_set:nn { teachingtools } {
      reset-example = section,
      prefix-separator = dash,
    }
  },
  example-prefix / subsection .code:n = {
    \tl_set:Nn \l_@@_example_prefix_tl { \arabic{subsection} }
    \bool_set_true:N \l_@@_example_prefix_bool
    \keys_set:nn { teachingtools } {
      reset-example = subsection,
      prefix-separator = dash,
    }
  },
  example-prefix / subsubsection .code:n = {
    \tl_set:Nn \l_@@_example_prefix_tl { \arabic{subsubsection} }
    \bool_set_true:N \l_@@_example_prefix_bool
    \keys_set:nn { teachingtools } {
      reset-example = subsubsection,
      prefix-separator = dash,
    }
  },
  example-prefix / paragraph .code:n = {
    \tl_set:Nn \l_@@_example_prefix_tl { \arabic{paragraph} }
    \bool_set_true:N \l_@@_example_prefix_bool
    \keys_set:nn { teachingtools } {
      reset-example = paragraph,
      prefix-separator = dash,
    }
  },
  example-prefix / subparagraph .code:n = {
    \tl_set:Nn \l_@@_example_prefix_tl { \arabic{subparagraph} }
    \bool_set_true:N \l_@@_example_prefix_bool
    \keys_set:nn { teachingtools } {
      reset-example = subparagraph,
      prefix-separator = dash,
    }
  },
  example-prefix / none .code:n = {
    \tl_gclear:N \l_@@_example_prefix_tl 
    \bool_set_false:N \l_@@_example_prefix_bool
  },
  example-prefix .default:n = none,
  example-prefix .initial:n = none,
}
%    \end{macrocode}
%\end{macro}
%
%\begin{macro}{
%  \l_@@_question_label_tl ,
%}
% The label on the question environment can be set using this key.  The default
% is Arabic numerals, and the options \opt{alph} and \opt{roman} are aliases for
% \opt{lower-alph} and \opt{lower-roman}.
%    \begin{macrocode}
\tl_new:N \l_@@_question_label_tl
\keys_define:nn { teachingtools } {
  question-label .choice:,
  question-label / arabic .code:n = {
    \tl_set:Nn \l_@@_question_label_tl { \arabic{question} }
  },
  question-label / alph .code:n = {
    \tl_set:Nn \l_@@_question_label_tl { \alph{question} }
  },
  question-label / lower-alph .code:n = {
    \tl_set:Nn \l_@@_question_label_tl { \alph{question} }
  },
  question-label / upper-alph .code:n = {
    \tl_set:Nn \l_@@_question_label_tl { \Alph{question} }
  },
  question-label / roman .code:n = {
    \tl_set:Nn \l_@@_question_label_tl { \roman{question} }
  },
  question-label / lower-roman .code:n = {
    \tl_set:Nn \l_@@_question_label_tl { \roman{question} }
  },
  question-label / upper-roman .code:n = {
    \tl_set:Nn \l_@@_question_label_tl { \Roman{question} }
  },
  question-label / footnote-symbol .code:n = {
    \tl_set:Nn \l_@@_question_label_tl { \fnsymbol{question} }
  },
  question-label .default:n = arabic,
  question-label .initial:n = arabic,
}
%    \end{macrocode}
%\end{macro}
%
%\begin{macro}{
%  \l_@@_example_label_tl
%}
% The label on the example environment can be set using this key.  The default
% is also Arabic numerals.
%    \begin{macrocode}
\tl_new:N \l_@@_example_label_tl
\keys_define:nn { teachingtools } {
  example-label .choice:,
  example-label / arabic .code:n = {
    \tl_set:Nn \l_@@_example_label_tl { \arabic{example} }
  },
  example-label / alph .code:n = {
    \tl_set:Nn \l_@@_example_label_tl { \alph{example} }
  },
  example-label / lower-alph .code:n = {
    \tl_set:Nn \l_@@_example_label_tl { \alph{example} }
  },
  example-label / upper-alph .code:n = {
    \tl_set:Nn \l_@@_example_label_tl { \Alph{example} }
  },
  example-label / roman .code:n = {
    \tl_set:Nn \l_@@_example_label_tl { \roman{example} }
  },
  example-label / lower-roman .code:n = {
    \tl_set:Nn \l_@@_example_label_tl { \roman{example} }
  },
  example-label / upper-roman .code:n = {
    \tl_set:Nn \l_@@_example_label_tl { \Roman{example} }
  },
  example-label / footnote-symbol .code:n = {
    \tl_set:Nn \l_@@_example_label_tl { \fnsymbol{example} }
  },
  example-label .default:n = arabic,
  example-label .initial:n = arabic,
}
%    \end{macrocode}
%\end{macro}
%
% \subsection{Dimensions and lengths}
% A number of dimensions and lengths are used internally to define spacing
% around each environment.  Each environment has its own skip length, initially
% set to the value of that for the \texttt{question} environment.
%    \begin{macrocode}
\skip_new:N \questionbefore
\skip_new:N \questionafter
\skip_new:N \answerbefore
\skip_new:N \answerafter
\skip_new:N \examplebefore
\skip_new:N \exampleafter
\skip_new:N \noticebefore
\skip_new:N \noticeafter

\skip_set:Nn \questionbefore { 1.0ex plus -1ex minus -0.25ex }
\skip_set:Nn \questionafter { 1ex plus 0.25ex }

\skip_set_eq:NN \answerbefore \questionbefore
\skip_set_eq:NN \examplebefore \questionbefore
\skip_set_eq:NN \noticebefore \questionbefore

\skip_set_eq:NN \answerafter \questionafter
\skip_set_eq:NN \exampleafter \questionafter
\skip_set_eq:NN \noticeafter \questionafter
%    \end{macrocode}
%
% \subsection{Environments}
% This package defines a number of environments for teaching materials.  
%
%\begin{environment}{question}
% The \cs{question} environment is used to create a list of questions.  The
% starred version is unnumbered and does not increment the question counter.
%    \begin{macrocode}
\NewDocumentEnvironment { question } { s } {
  \par \skip_vertical:N \questionbefore
  \IfBooleanTF #1 
    { { \bfseries Question.\par } }
    {
      \refstepcounter{question}
      {\bfseries Question~\@@_print_question: .\par}
    }
  \begin{itshape}
}
{\end{itshape}\par \skip_vertical:N \questionafter}
\cs_new:cpn {question*} {\question*}
\cs_new_eq:cN {endquestion*} \endquestion
%    \end{macrocode}
%\end{environment}
%
%\begin{environment}{example}
% The \cs{example} environment is used to create a list of examples.  The
% starred version is unnumbered and does not increment the example counter.
%    \begin{macrocode}
\NewDocumentEnvironment { example } { s } {
  \par\vspace\examplebefore
  \IfBooleanTF #1 
    { { \bfseries Example.\par } }
    {
      \refstepcounter{example}
      {\bfseries Example~\@@_print_example: .\par}
    }
}
{\par\vspace\exampleafter}
\cs_new:cpn {example*} {\example*}
\cs_new_eq:cN {endexample*} \endexample
%    \end{macrocode}
%\end{environment}
%
% \subsection{Document macros}
%
% The user document macros are all collected together here for ease.
%
%\begin{macro}{\ttsetup}
% The set-up macro simply moves to the correct path and executes
% whatever has been passed.
%    \begin{macrocode}
\NewDocumentCommand \ttsetup { m } {
  \keys_set:nn { teachingtools } {#1}
}
%    \end{macrocode}
%\end{macro}
%
% \subsection{Loading configuration}
%
% Apply the settings given at load time.
%    \begin{macrocode}
\ProcessKeysOptions { teachingtools }
%    \end{macrocode}
%
%    \begin{macrocode}
%</package>
%    \end{macrocode}
%
% \end{implementation}
%
% \PrintChanges
%
% \newpage
%
% \PrintIndex
%
\endinput
