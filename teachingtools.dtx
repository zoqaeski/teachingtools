% \iffalse meta-comment
%<*internal>
\iffalse
%</internal>
%<*readme>
teachingtools - Environments and tools for teaching materials
===============================================================

Teaching materials, such as lecture notes, tutorials, and textbooks often
require worked examples and questions for students to learn from.  This package
provides a set of environments suitable to typeset question/answer pairs,
examples, and notes.

Installation
------------

The package is supplied in `dtx` format, and built with `l3build` to take
advantage of new LaTeX3 and LuaTeX features. 

The package requires LaTeX3 support as provided in the `l3kernel` and
`l3packages` bundles. Both of these are available on
[CTAN](http://www.ctan.org/) as ready-to-install zip files.  Suitable versions
are available in MiKTeX 2.9 and TeX Live 2015 (updating the relevant packages
online may be necessary).  LaTeX3, and so `teachingtools`, requires the e-TeX
extensions: these are available on all modern TeX systems.
%</readme>
%<*internal>
\fi
\def\nameofplainTeX{plain}
\ifx\fmtname\nameofplainTeX\else
  \expandafter\begingroup
\fi
%</internal>
%<*install>
\input l3docstrip.tex
\keepsilent
\askforoverwritefalse
\preamble
---------------------------------------------------------------
The teachingtools package --- Environments and tools for teaching materials
Maintained by Robbie Smith
E-mail: zoqaeski@gmail.com
Released under the LaTeX Project Public License v1.3c or later
See http://www.latex-project.org/lppl.txt
---------------------------------------------------------------

\endpreamble
\postamble
Copyright (C) 2017 by
  Robbie Smith <zoqaeski@gmail.com>

It may be distributed and/or modified under the conditions of
the LaTeX Project Public License (LPPL), either version 1.3c of
this license or (at your option) any later version.  The latest
version of this license is in the file:
   http://www.latex-project.org/lppl.txt

This work is "maintained" (as per LPPL maintenance status) by
  Robbie Smith.

This work consists of the file  teachingtools.dtx
          and the derived files teachingtools.pdf,
                                teachingtools.sty and
                                teachingtools.ins.
\endpostamble
\usedir{tex/latex/teachingtools}
\generate{
  \file{\jobname.sty}{\from{\jobname.dtx}{package}}
}
%</install>
%<install>\endbatchfile
%<*internal>
\usedir{source/latex/teachingtools}
\generate{
  \file{\jobname.ins}{\from{\jobname.dtx}{install}}
}
\nopreamble\nopostamble
\usedir{doc/latex/teachingtools}
\generate{
  \file{README.md}{\from{\jobname.dtx}{readme}}
}
\ifx\fmtname\nameofplainTeX
  \expandafter\endbatchfile
\else
  \expandafter\endgroup
\fi
%</internal>
%<*driver|package>
\RequirePackage{expl3}[2015/09/11]
\RequirePackage{xparse}
%</driver|package>
%<*driver>
\documentclass{l3doc}
%\DisableImplementation
\begin{document}
  \DocInput{\jobname.dtx}
\end{document}
%</driver>
% \fi
%\ProvideDocumentCommand\opt{m}{\texttt{#1}}
%
%\GetFileInfo{\jobname.sty}
%
%\title{^^A
%  \textsf{teachingtools} --- Environments and tools for teaching materials \thanks{^^A
%    This file describes \fileversion, last revised \filedate.^^A
%  }^^A
%}
%\author{^^A
%  Robbie Smith\thanks
%    {^^A
%      E-mail: \href{mailto:zoqaeski@gmail.com}
%      {\texttt{zoqaeski@gmail.com}}^^A
%    }^^A
%}
%\date{Released \filedate}
%
%\maketitle
%
%\changes{v0.1}{2017/11}{First public testing release}
%
%\begin{implementation}
%
%\section{Implementation}
%
%    \begin{macrocode}
%<*package>
%    \end{macrocode}
%
%    \begin{macrocode}
%<@@=teachingtools>
%    \end{macrocode}
%
%\subsection{Preliminaries}
%
% The usual preliminaries.
%    \begin{macrocode}
\ProvidesExplPackage {teachingtools} {2017/11/18} {0.1}
  {Environments and tools for teaching materials}
%    \end{macrocode}
%
% This package depends on recent versions of the \pkg{l3kernel}. 
%    \begin{macrocode}
\@ifpackagelater { expl3 } { 2015/11/15 }
  { }
  {
    \PackageError { teachingtools } { Support~package~expl3~too~old }
      {
        You~need~to~update~your~installation~of~the~bundles~'l3kernel'~and~
        'l3packages'.\MessageBreak
        Loading~teachingtools~will~abort!
      }
    \tex_endinput:D
  }
%    \end{macrocode}
% Now load the support packages. \pkg{chngcntr} is required to modify counters
% while they're in use, and \pkg{enumitem} provides convenience macros for
% modifing lists.
%    \begin{macrocode}
\RequirePackage{ l3keys2e , chngcntr , enumitem }
%    \end{macrocode}
%
%\begin{macro}{
%  \@@_error:nxx ,
%  \@@_error:nx  ,
%  \@@_error:n
%}
%\begin{macro}{\l_@@_error_bool}
% Some general-purpose functions for throwing errors are defined next.
%
%    \begin{macrocode}
\cs_new_protected:Npn \@@_error:nxx #1#2#3 {
  \bool_set_true:N \l_@@_error_bool
  \msg_error:nnxx { teachingtools } {#1} {#2} {#3}
}
\cs_new_protected:Npn \@@_error:nx #1#2 {
  \@@_error:nxx {#1} {#2} { }
}
\cs_new_protected:Npn \@@_error:n #1 {
  \@@_error:nxx {#1} { } { }
}
\bool_new:N \l_@@_error_bool
%    \end{macrocode}
%\end{macro}
%\end{macro}
%
%\begin{macro}{\l_@@_key_tl}
% Unknown options should raise an \pkg{teachingtools}-specific error message.
%    \begin{macrocode}
\tl_new:N  \l_@@_key_tl
\keys_define:nn { teachingtools } {
  unknown .code:n = {
    \msg_error:nnx { teachingtools } { unknown-option }
    { \exp_not:V \l_keys_key_tl }
  }
}
\AtBeginDocument {
  \keys_define:nn { teachingtools } {
    unknown .code:n = {
      \msg_error:nnx { teachingtools } { unknown-option }
      { \exp_not:V \l_keys_key_tl }
    }
  }
}
%    \end{macrocode}
%\end{macro}
%
% \subsection{Options}
% The following options are provided to control the behaviour of
% \pkg{teachingtools}.
%
%\begin{macro}{
%  \l_@@_reset_all_tl ,
%  \l_@@_reset_question_tl        ,
%  \l_@@_reset_example_tl         ,
%}
% The \opt{reset-all} option provides a means for environment numbering to be reset
% at different sectioning \enquote{levels}.  The default is for numbering to be
% set individually for each environment.
%   \begin{macrocode}
% \keys_define:nn { teachingtools } {
%   reset-all .choice:,
%   reset-all .default:n = false,
%   reset-all / part .meta:n = {
%     reset-question = part,
%     reset-example = part
%   },
%   reset-all / chapter .meta:n = {
%     reset-question = chapter,
%     reset-example = chapter
%   },
%   reset-all / section .meta:n = {
%     reset-question = section,
%     reset-example = section
%   },
%   reset-all / subsection .meta:n = {
%     reset-question = subsection,
%     reset-example = subsection
%   },
%   reset-all / subsubsection .meta:n = {
%     reset-question = subsubsection,
%     reset-example = subsubsection
%   },
%   reset-all / paragraph .meta:n = {
%     reset-question = paragraph,
%     reset-example = paragraph
%   },
%   reset-all / subparagraph .meta:n = {
%     reset-question = subparagraph,
%     reset-example = subparagraph
%   },
%   reset-all / false .meta:n = {
%     reset-question = section,
%     reset-example = none
%   }
% }
%   \end{macrocode}
%
% The \opt{reset-question} option provides a means for question numbering to be reset
% at different sectioning \enquote{levels}.  The default is for numbering to be
% reset at the \enquote{section} level.
%   \begin{macrocode}
\keys_define:nn { teachingtools } {
  reset-question .default:n = section,
  reset-question .choices:nn = {
    part, chapter, section, subsection, subsubsection, paragraph, subparagraph
  }
  {
    \counterwithin*{question}{\l_keys_choice_tl}
  },
  reset-question / none .code:n = {
    \counterwithout*{question}{section}
  },
}
%   \end{macrocode}
%
% The \opt{reset-example} option provides a means for example numbering to be reset
% at different sectioning \enquote{levels}.  The default is for numbering to be
% not reset.
%   \begin{macrocode}
\keys_define:nn { teachingtools } {
  reset-example .default:n = none,
  reset-example .choices:nn = {
    part, chapter, section, subsection, subsubsection, paragraph, subparagraph
  }
  {
    \counterwithin*{example}{\l_keys_choice_tl}
  },
  reset-example / none .code:n = {
    \counterwithout*{example}{section}
  },
}
%   \end{macrocode}
%\end{macro}
%\end{macro}



%
%\subsection{Messages}
% All of the error and warning messages are defined here in one place to
% simplify things.
%
%    \begin{macrocode}
\msg_new:nnnn { teachingtools } { unknown-option }
  { Unknown~option~'#1'. }
  {
    The~option~'#1'~is~not~known~by~teachingtools:
    perhaps~it~is~spelled~incorrectly.
  }
%    \end{macrocode}
%
% \subsection{Dimensions and Counters}
% A number of dimensions and lengths are used internally to define spacing
% around each environment, and counters are used to ensure that the environments
% can be sequentially numbered.
%
% Each environment has its own skip length, initially set to the value of that
% for the \cs{question} environment.
%   \begin{macrocode}
\skip_new:N \questionbefore
\skip_new:N \questionafter
\skip_new:N \answerbefore
\skip_new:N \answerafter
\skip_new:N \examplebefore
\skip_new:N \exampleafter
\skip_new:N \noticebefore
\skip_new:N \noticeafter
%
\skip_set:Nn \questionbefore { 1.0ex plus -1ex minus -0.25ex }
\skip_set:Nn \questionafter { 1ex plus 0.25ex }
%
\skip_set_eq:NN \answerbefore \questionbefore
\skip_set_eq:NN \examplebefore \questionbefore
\skip_set_eq:NN \noticebefore \questionbefore
%
\skip_set_eq:NN \answerafter \questionafter
\skip_set_eq:NN \exampleafter \questionafter
\skip_set_eq:NN \noticeafter \questionafter
%   \end{macrocode}
%
% Each environment also has its own counter, which can be set and incremented
% individually.
%   \begin{macrocode}
\newcounter{question}
\newcounter{example}
%   \end{macrocode}
%
% \subsection{Environments}
% This package defines a number of environments for teaching materials.  
%

\addtotoclist{loq}
\newcommand{\listofloqname}{List~of~Questions}
\newcommand{\listofquestions}{\listoftoc{loq}}
\setuptoc{loq}{leveldown}

%\begin{environment}{question}
% The \cs{question} environment is used to create a list of questions.  The
% starred version is unnumbered and does not increment the question counter.
%   \begin{macrocode}
\NewDocumentEnvironment { question } { s } {
  \par\vspace\questionbefore
  \IfBooleanTF #1 
    { { \bfseries Question.\par } }
    {
      \refstepcounter{question}
      {\bfseries Question~\thequestion.\par}
    }
  \begin{itshape}
}
{\end{itshape}\par\vspace\questionafter}
\cs_new:cpn {question*} {\question*}
\cs_new_eq:cN {endquestion*} \endquestion
%   \end{macrocode}
%\end{environment}
%
%\begin{environment}{example}
% The \cs{example} environment is used to create a list of examples.  The
% starred version is unnumbered and does not increment the example counter.
%   \begin{macrocode}
\NewDocumentEnvironment { example } { s } {
  \par\vspace\examplebefore
  \IfBooleanTF #1 
    { { \bfseries Example.\par } }
    {
      \refstepcounter{example}
      {\bfseries Example~\theexample.\par}
    }
}
{\par\vspace\exampleafter}
\cs_new:cpn {example*} {\example*}
\cs_new_eq:cN {endexample*} \endexample
%   \end{macrocode}
%\end{environment}
%
% \subsection{Document macros}
%
% The user document macros are all collected together here for ease.
%
%\begin{macro}{\ttsetup}
% The set-up macro simply moves to the correct path and executes
% whatever has been passed.
%    \begin{macrocode}
\NewDocumentCommand \ttsetup { m } {
  \keys_set:nn { teachingtools } {#1}
}
%    \end{macrocode}
%\end{macro}
%
% \subsection{Loading configuration}
%
% Apply the settings given at load time.
%    \begin{macrocode}
\ProcessKeysOptions { teachingtools }
%    \end{macrocode}
%
%    \begin{macrocode}
%</package>
%    \end{macrocode}
%
% \end{implementation}
\endinput
