% \iffalse meta-comment
%<*internal>
\iffalse
%</internal>
%<*readme>
teachingtools - Environments and tools for teaching materials
===============================================================

Teaching materials, such as lecture notes, tutorials, and textbooks often
require worked examples and questions for students to learn from.  This package
provides a set of environments suitable to typeset question/answer pairs,
examples, and notes.

Installation
------------

The package is supplied in `dtx` format, and built with `l3build` to take
advantage of new LaTeX3 and LuaTeX features. 

The package requires LaTeX3 support as provided in the `l3kernel` and
`l3packages` bundles. Both of these are available on
[CTAN](http://www.ctan.org/) as ready-to-install zip files.  Suitable versions
are available in MiKTeX 2.9 and TeX Live 2015 (updating the relevant packages
online may be necessary).  LaTeX3, and so `teachingtools`, requires the e-TeX
extensions: these are available on all modern TeX systems.
%</readme>
%<*internal>
\fi
\def\nameofplainTeX{plain}
\ifx\fmtname\nameofplainTeX\else
  \expandafter\begingroup
\fi
%</internal>
%<*install>
\input l3docstrip.tex
\keepsilent
\askforoverwritefalse
\preamble
---------------------------------------------------------------
The teachingtools package --- Environments and tools for teaching materials
Maintained by Robbie Smith
E-mail: zoqaeski AT gmail DOT com
Released under the LaTeX Project Public License v1.3c or later
See http://www.latex-project.org/lppl.txt
---------------------------------------------------------------

\endpreamble
\postamble
Copyright (C) 2017 by
  Robbie Smith <zoqaeski AT gmail DOT com>

It may be distributed and/or modified under the conditions of
the LaTeX Project Public License (LPPL), either version 1.3c of
this license or (at your option) any later version.  The latest
version of this license is in the file:
   http://www.latex-project.org/lppl.txt

This work is "maintained" (as per LPPL maintenance status) by
  Robbie Smith.

This work consists of the file  teachingtools.dtx
          and the derived files teachingtools.pdf,
                                teachingtools.sty and
                                teachingtools.ins.
\endpostamble
\usedir{tex/latex/teachingtools}
\generate{
  \file{\jobname.sty}{\from{\jobname.dtx}{package}}
}
%</install>
%<install>\endbatchfile
%<*internal>
\usedir{source/latex/siunitx}
\generate{
  \file{\jobname.ins}{\from{\jobname.dtx}{install}}
}
\nopreamble\nopostamble
\usedir{doc/latex/siunitx}
\generate{
  \file{README.md}{\from{\jobname.dtx}{readme}}
}
\ifx\fmtname\nameofplainTeX
  \expandafter\endbatchfile
\else
  \expandafter\endgroup
\fi
%</internal>
%<*driver|package>
\RequirePackage{expl3}[2015/09/11]
\RequirePackage{xparse}
%</driver|package>
%<*driver>
\documentclass{l3doc}
%\DisableImplementation
\begin{document}
  \DocInput{\jobname.dtx}
\end{document}
%</driver>
% \fi
%\begin{implementation}
%
%\section{Implementation}
%
%    \begin{macrocode}
%<*package>
%    \end{macrocode}
%
%    \begin{macrocode}
%<@@=teachingtools>
%    \end{macrocode}
%
%\subsection{Preliminaries}
%
% The usual preliminaries.
%    \begin{macrocode}
\ProvidesExplPackage {teachingtools} {2017/11/18} {0.1}
  {Environments and tools for teaching materials}
%    \end{macrocode}
%
% This package depends on recent versions of the \pkg{l3kernel}. 
%    \begin{macrocode}
\@ifpackagelater { expl3 } { 2015/11/15 }
  { }
  {
    \PackageError { teachingtools } { Support~package~expl3~too~old }
      {
        You~need~to~update~your~installation~of~the~bundles~'l3kernel'~and~
        'l3packages'.\MessageBreak
        Loading~teachingtools~will~abort!
      }
    \tex_endinput:D
  }
%    \end{macrocode}
% Now load the support packages. \pkg{chngcntr} is required to modify counters
% while they're in use, and \pkg{enumitem} provides convenience macros for
% modifing lists.
%    \begin{macrocode}
\RequirePackage{ l3keys2e , chngcntr , enumitem }
%    \end{macrocode}
%
% This package defines a few environments, of which 



\keys_define:nn { teachingtools } {
  reset-at .choice:,
  reset-at .default:n = section,
  reset-at / part .code:n = {
    \counterwithin*{question}{part}
  },
  reset-at / chapter .code:n = {
    \counterwithin*{question}{chapter}
  },
  reset-at / section .code:n = {
    \counterwithin*{question}{section}
  },
  reset-at / subsection .code:n = {
    \counterwithin*{question}{subsection}
  },
  reset-at / subsubsection .code:n = {
    \counterwithin*{question}{subsubsection}
  },
  reset-at / paragraph .code:n = {
    \counterwithin*{question}{paragraph}
  },
  reset-at / subparagraph .code:n = {
    \counterwithin*{question}{subparagraph}
  },
  reset-at / none .code:n = {
    \counterwithout*{question}{section}
  },

}

\newcounter{question}
\newcounter{example}

\newlength{\questionbeforeskip}
\newlength{\questionafterskip}
\setlength{\questionbeforeskip}{1.0ex plus -1ex minus -0.25ex}
\setlength{\questionafterskip}{1ex plus 0.25ex}
\newlength{\answerskip}
\setlength{\answerskip}{0.25ex plus -0.125ex minus -0.125ex}

\addtotoclist{loq}
\newcommand{\listofloqname}{List~of~Questions}
\newcommand{\listofquestions}{\listoftoc{loq}}
\setuptoc{loq}{leveldown}

\NewDocumentEnvironment { question } { s } {
  \refstepcounter{question}
  {\par\vspace\questionbeforeskip{\bfseries Question~\thequestion.}\par}
  \begin{itshape}
}
{\end{itshape}\par\vspace\questionafterskip}

%\begin{macro}{\ttsetup}
% The set up macro simply moves to the correct path and executes
% whatever has been passed.
%    \begin{macrocode}
\NewDocumentCommand \ttsetup { m } {
  \keys_set:nn { teachingtools } {#1}
}
%    \end{macrocode}
%\end{macro}

\ProcessKeysOptions { teachingtools }
\endinput
