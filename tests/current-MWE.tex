    \documentclass{article}
    \usepackage{xparse}
    \usepackage{expl3}
    \usepackage{l3keys2e}
    \usepackage{chngcntr}

    \setlength\parindent{0pt}

    \ExplSyntaxOn

    \keys_define:nn { teachingtools } {
      reset-all .choice:,
      reset-all .default:n = false,
      reset-all / part .meta:n = {
        reset-question = part,
        reset-example = part
      },
      reset-all / chapter .meta:n = {
        reset-question = chapter,
        reset-example = chapter
      },
      reset-all / section .meta:n = {
        reset-question = section,
        reset-example = section
      },
      reset-all / subsection .meta:n = {
        reset-question = subsection,
        reset-example = subsection
      },
      reset-all / subsubsection .meta:n = {
        reset-question = subsubsection,
        reset-example = subsubsection
      },
      reset-all / paragraph .meta:n = {
        reset-question = paragraph,
        reset-example = paragraph
      },
      reset-all / subparagraph .meta:n = {
        reset-question = subparagraph,
        reset-example = subparagraph
      },
      reset-all / false .meta:n = {
        reset-question = section,
        reset-example = none
      }
    }
    \keys_define:nn { teachingtools } {
      reset-question .default:n = section,
      reset-question .choices:nn = {
        part, chapter, section, subsection, subsubsection, paragraph, subparagraph
      }
      {
        \counterwithin*{question}{\l_keys_choice_tl}
      },
      reset-question / none .code:n = {
        \counterwithout*{question}{section}
      },
    }
    \keys_define:nn { teachingtools } {
      reset-example .default:n = none,
      reset-example .choices:nn = {
        part, chapter, section, subsection, subsubsection, paragraph, subparagraph
      }
      {
        \counterwithin*{example}{\l_keys_choice_tl}
      },
      reset-example / none .code:n = {
        \counterwithout*{example}{section}
      },
    }

    \newcounter{question}
    \newcounter{example}

    \NewDocumentEnvironment { question } { s } {
      \par
      \IfBooleanTF #1
        { { \bfseries Question.\par } }
        {
          \refstepcounter{question}
          {\bfseries Question~\thequestion.\par}
        }
      \begin{itshape}
    }
    {\end{itshape}\par}
    \cs_new:cpn {question*} {\question*}
    \cs_new_eq:cN {endquestion*} \endquestion
    \NewDocumentEnvironment { example } { s } {
      \par
      \IfBooleanTF #1
        { { \bfseries Example.\par } }
        {
          \refstepcounter{example}
          {\bfseries Example~\theexample.\par}
        }
    }
    {\par}
    \cs_new:cpn {example*} {\example*}
    \cs_new_eq:cN {endexample*} \endexample
    \NewDocumentCommand \ttsetup { m } {
      \keys_set:nn { teachingtools } {#1}
    }
    \ExplSyntaxOff

    \begin{document}

    \ttsetup{
      % reset-all = false,
      % reset-question = subsection,
      % reset-example = section,
    }

    \section{A section}

    \begin{question}
    The question in a section
    \end{question}

    \begin{example}
        This is an example of the \texttt{example} environment.
    \end{example}

    \subsection{A subsection}

    \begin{question}
    The question in a subsection
    \end{question}

    \begin{example*}
        This is an example of the \texttt{example*} environment.
    \end{example*}

    \begin{question}
    Another question in a subsection
    \end{question}

    \begin{example}
        This is an example of the \texttt{example} environment.
    \end{example}

    \subsubsection{A subsubsection}

    \begin{question*}
    A starred question in a subsubsection
    \end{question*}

    \begin{example}
        This is an example of the \texttt{example} environment.
    \end{example}


    \begin{question}
    The question in a subsubsection
    \end{question}

    \begin{question}
    Another question in a subsubsection
    \end{question}

    \begin{example}
        This is an example of the \texttt{example} environment.
    \end{example}

    \subsubsection{A subsubsection}

    \begin{question}
    The question in a subsubsection
    \end{question}

    \begin{example}
        This is an example of the \texttt{example} environment.
    \end{example}

    \begin{question}
    Another question in a subsubsection
    \end{question}

    \subsection{Another subsection}

    \begin{example}
        This is an example of the \texttt{example} environment.
    \end{example}

    \begin{question}
    Another question in a subsection
    \end{question}

    \begin{question}
    What is this question?
    \end{question}

    \section{Another section}

    \begin{question}
    Another question in another section
    \end{question}

    \begin{question}
    Another question in another section
    \end{question}

    \begin{example}
        This is an example of the \texttt{example} environment.
    \end{example}

    \subsection{A subsection}

    \begin{question}
    The question in a subsection
    \end{question}

    \begin{question}
    Another question in a subsection
    \end{question}

    \begin{example}
        This is an example of the \texttt{example} environment.
    \end{example}

    \begin{question}
    Another question in a subsection
    \end{question}

    \subsubsection{A subsubsection}

    \begin{question}
    The question in a subsubsection
    \end{question}

    \begin{question}
    Another question in a subsubsection
    \end{question}

    \begin{example}
        This is an example of the \texttt{example} environment.
    \end{example}

    \subsubsection{A subsubsection}

    \begin{question}
    The question in a subsubsection
    \end{question}

    \begin{question}
    Another question in a subsubsection
    \end{question}

    \begin{example}
        This is an example of the \texttt{example} environment.
    \end{example}

    \end{document}
